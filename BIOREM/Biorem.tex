\documentclass[11pt]{article} 
\usepackage[latin1]{inputenc}
\usepackage[frenchb]{babel}

\usepackage{geometry} 
\geometry{a4paper}
%\geometry{hscale=0.7,vscale=0.7,centering}

\usepackage{graphicx} 

\usepackage{float} % Allows putting an [H] in \begin{figure} to specify the exact location of the figure
\usepackage{wrapfig} % Allows in-line images such as the example fish picture

\linespread{1.2} 

\setlength\parindent{0pt} % Uncomment to remove all indentation from paragraphs

\graphicspath{{Pictures/}} % Specifies the directory where pictures are stored
\usepackage[absolute]{textpos}
\usepackage{array}
\usepackage {tabularx}
\usepackage {booktabs}
\usepackage {colortbl}
\usepackage {float}
\usepackage {fancyhdr}
\usepackage{acronym}
\usepackage{pdfpages}
\usepackage{SIunits}
\usepackage{natbib}
\usepackage{hyperref}

\setcitestyle{square}
\bibliographystyle{apalike}
\begin{document}

%----------------------------------------------------------------------------------------
%	TITLE PAGE
%----------------------------------------------------------------------------------------

\begin{titlepage}
\begin{center}
\begin{figure}[H]
\includegraphics[width= \textwidth]{ULB.jpeg}
\end{figure}
\end{center}

\newcommand{\HRule}{\rule{\linewidth}{0.5mm}} % Defines a new command for the horizontal lines, change thickness here

\center % Center everything on the page

\vspace*{\stretch{10}} %centrer verticalement

\textsc{\LARGE Universit� Libre de Bruxelles}\\[1.5cm] 
\textsc{\Large Laboratoires BIOMAR et ESA}\\[0.5cm] 
\textsc{\large BING-F531}\\[0.5cm] 

\HRule \\[0.4cm]
{ \huge \bfseries Recherche bibliographique sur le plastique et ses voies de biod�gradation }\\[0.4cm] 
\HRule \\[1.5cm]

\begin{minipage}{0.4\textwidth}
\begin{flushleft} \large
\emph{Auteurs:}\\
Rafael \textsc{Colomer Martinez}\\
Nicolas \textsc{Piret}
 \end{flushleft}
\end{minipage}
~
\begin{minipage}{0.4\textwidth}
\begin{flushright} \large
\emph{Professeur:} \\
Dr. Isabelle \textsc{George} \\
\end{flushright}
\end{minipage}\\[4cm]

\vspace*{\stretch{10}} %centrer verticalement


{\large \today}\\[3cm] 

\vfill % Fill the rest of the page with whitespace

\end{titlepage}

%----------------------------------------------------------------------------------------
%	TABLE OF CONTENTS
%----------------------------------------------------------------------------------------

\tableofcontents % Include a table of contents

\newpage % Begins the essay on a new page instead of on the same page as the table of contents 


%----------------------------------------------------------------------------------------
%	ACRONYMES
%----------------------------------------------------------------------------------------

\section*{Liste des abr�viations et acronymes}%section sans num�ro
\begin{acronym}[SDS-PAGE]%mettre le plus long acronyme ici
\acro{BCA}{\emph{ou Pierce BCA - M�thode de dosage prot�ique colorim�trique bas�e sur l'emploi d'acide bicinchonique}}
\acro{BPF/GMP}{\emph{Bonnes pratiques de fabrication ou Good Manufacturing Practices - Notion d'assurance de qualit�}}
\acro{CDU}{\emph{Casein Digestion Unit Analytical Method - M�thode spectroscopique de d�tection d'acides amin�s issus d'une d�gradation enzymatique sur substrat de cas�ine}}
\acro{CV}{\emph{Curriculum Vitae}}
\acro{PI}{\emph{Point iso�lectrique}}
\acro{QC}{\emph{Contr�le qualit�}}
\acro{SDS-PAGE}{\emph{Sodium Dod�cyl Sulfate Polyacrylamide Gel Electrophoresis}}
\acro{TU}{\emph{Tyrosine Unit Analytical Method - M�thode spectroscopique de d�tection d'acides amin�s issus d'une d�gradation enzymatique sur substrat de cas�ine}}
\acro{UF}{\emph{Ultrafiltration}}
\end{acronym}
\addcontentsline{toc}{section}{Liste des abr�viations et acronymes}%mettre dans la table des mat.
\clearpage

%----------------------------------------------------------------------------------------
%	DOCUMENT
%----------------------------------------------------------------------------------------

\section{Introduction}

Tu sens le caca

\section{Description du plastique}
\section{Utilisation, production, sources de pollution et types d'environnements contamin�s}
\section{Dangers potentiels pour l'environnement}
\section{Aspects l�gislatifs}
\section{Bior�m�diation des milieux contamin�s}

blabla \citep{Mills2009}
blabla \citep{friedman1964kinetics}


\bibliography{Bibliography/Plastics}

\end{document}
