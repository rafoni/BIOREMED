\subsection{D\'egradation abiotique des plastiques}
\subsubsection{Photo-d\'egradation}
\par{
Dans les conditions ambiantes, la photo-d\'egradation est l'un des principaux moyens par lesquels les mati�res plastiques  sont d\'egrad\'ees. Les principaux processus impliqu\'es sont les coupures de cha�nes au cours du photo-vieillissement {\citep{ramone2015evolutions}} lorsqu'elles sont expos\'ees � un rayonnement ultraviolet (UV) (290-400 nm) ou � un rayonnement visible (400-700 nm). La longueur d'onde UV la plus dommageable pour un mat\'eriau sp\'ecifique d\'epend des liaisons pr\'esentes. Pour le poly\'ethyl�ne, il s'agit de 300 nm et pour le polypropyl�ne de  370 nm {\citep{lambert2014occurrence}}.
}

\subsubsection{D\'egradation thermique}
\par{
La d\'egradation thermique est la d\'et\'erioration mol\'eculaire d'un polym�re � la suite d'une augmentation de temp\'erature provoquant la scission de la cha�ne polym�re principale. Des changements de propri\'et\'es affectent le polym�re dans son enti�ret\'e tant au niveau du poids mol\'eculaire que de ses propri\'et\'es de r\'esistance, cristallinit\'e, fragilisation ou craquage {\citep{lambert2014occurrence}}.
}

\subsubsection{D�gradation par oxydation}
\par{
Les proc\'ed\'es d'oxydation peuvent �tre photo ou thermiquement induits et sont consid\'er\'es comme importants, en particulier pour les mat\'eriaux non hydrolysables {\citep{lambert2014occurrence}}.
}

\subsubsection{Hydrolyse}
\par{
Le  taux d'hydrolyse d\'epend de la pr\'esence de liaisons covalentes hydrolysables telles que les groupes ester, \'ether, anhydride, amide, carbamide (ur\'ee) ou ester amide (ur\'ethanne) dans le polym�re {\citep{lambert2014occurrence}}. La premi�re \'etape de d\'egradation du plastique est l'hydrolyse des fonctions esters qui a lieu al\'eatoirement sur la cha�ne carbon\'ee. La cin\'etique de la r\'eaction d\'epend de la temp\'erature, de la morphologie du polym�re et de la pr\'esence d'acide ou de base {\citep{ramone2015evolutions}}. Cette r\'eaction entra�ne une alt\'eration des propri\'et\'es du plastique et sa r\'esistance.
}

\subsubsection{D\'esint\'egration m\'ecanique}
\par{
La rupture est une d\'esint\'egration m\'ecanique provoqu\'ee par des forces internes et externes accompagn\'ee ou non d'une d\'eformation {\citep{kausch2001materiaux}}. Ce proc\'ed\'e se distingue de la d\'egradation car les liaisons mol\'eculaires des mat\'eriaux restent inchang\'ees. Les plastiques sont expos\'es � diff\'erentes formes de d\'egradation m\'ecanique que ce soit suite aux conditions climatiques, aux mouvements de l'eau ou � des d\'eg�ts r\'ealis\'es par les animaux ou les oiseaux {\citep{lambert2014occurrence}}.
}
