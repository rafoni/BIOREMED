\section{Introduction}

\par{
La croissance de la population mondiale et l'\'evolution des modes de vie entrainent une augmentation des besoins. Une agriculture intensive, un d\'eveloppement industriel massif et un mode de vie moderne bas\'e sur la consommation en sont la cons\'equence. La production des d\'echets, issus de ces activit\'es est de plus en plus importante et leur rejet dans l'environnement contribue \`a une pollution du milieu. G\'en\'eralement, l'importance du polluant est li\'ee \`a sa toxicit\'e pour l'homme. D'autres polluants, non toxiques pour l'homme, sont \`a pr\'esent \'egalement consid\'er\'es comme polluants de par un effet indirect pour l'homme suite \`a son accumulation dans l'environnement. La pollution par le plastique, mat\'eriau polym\`ere, en est un exemple.
}\\
\par{
Le plastique est un symbole de la soci\'et\'e de consommation que l'on retrouve partout sous diff\'erentes formes dans la vie quotidienne et dans les industries. Principalement issu du p\'etrole, il a des propri\'et\'es chimiques et physiques qui en font un produit r\'esistant, l\'eger et souple,  facilement utilisable et rapidement jetable. Les polym\`eres plastiques sont rarement utilis\'es \`a l'\'etat brut  et g\'en\'eralement, les r\'esines polym\`eres sont m\'elang\'ees avec divers additifs pour am\'eliorer les performances. Son faible co\^ut de production lui donne un autre avantage et entraine une production et une utilisation massives.
}\\
\par{
Caus\'ee par l'accumulation de d\'echets de mati\`ere plastique sous forme de d\'echets solides, la pollution par les mati\`eres plastiques, en augmentation, pr\'esente un probl\`eme d'ordre \'ecologique pouvant avoir des r\'epercussions indirectes sur l'homme. Elle se compose de d\'echets sous forme de d\'ebris visibles que l'on appelle macroplastiques posant des probl\`emes pour la faune marine notamment. D'autres d\'echets compos\'es d'\'el\'ements plastiques de petite taille, soit fabriqu\'es de la sorte soit provenant de la d\'egradation de d\'echets de plus grande taille, d\'esign\'es sous le nom de microplastiques, causent maintenant d'importants dommages aux \'ecosyst\`emes aquatiques transf\'erant potentiellement des substances toxiques dans la cha\^ine alimentaire et pouvant donc affecter les grands animaux et les oiseaux et secondairement les hommes. Outre les effets n\'efastes sur les \'ecosyst\`emes aquatiques, certains additifs ajout\'es lors de la fabrication pourraient avoir un effet n\'efaste sur la sant\'e. 
}\\
\par{
Comme ces mat\'eriaux ne sont que difficilement biod\'egradables, afin d'\'eviter leur accumulation dans l'environnement, la r\'eutilisation et le recyclage sont des solutions actuellement propos\'ees pour r\'eduire l'impact de cette pollution. 
}\\
\par{
Une prise de conscience des cons\'equences n\'egatives des d\'echets plastiques a permis la mise en place de  l\'egislations en mati\`ere de fabrication et distribution de ces produits afin de limiter les effets ind\'esirables sur l'environnement et la sant\'e humaine.
}\\
\par{
Dans le cadre de ce travail, nous nous sommes int\'eress\'es \`a une recherche plus approfondie sur les techniques de biorem\'ediation faisant intervenir des microorganismes par l'action directe de d\'e
gradation naturelle. 
}
